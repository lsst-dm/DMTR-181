\documentclass[DM,lsstdraft,STR,toc]{lsstdoc}
\usepackage{geometry}
\usepackage{longtable,booktabs}
\usepackage{enumitem}
\usepackage{arydshln}

\input meta.tex

\providecommand{\tightlist}{
  \setlength{\itemsep}{0pt}\setlength{\parskip}{0pt}}

\begin{document}

\def\milestoneName{DAQ Validation}
\def\milestoneId{LDM-503-10}
\def\product{Data Management}

\setDocCompact{true}

\title{ LDM-503-10 DAQ Validation Test Plan and Report}
\setDocRef{\lsstDocType-\lsstDocNum}
\date{\vcsdate}
\setDocUpstreamLocation{\url{https://github.com/lsst/lsst-texmf/examples}}
\author{  }

\input history_and_info.tex


\setDocAbstract{
This is the test plan and report for LDM-503-10 (DAQ Validation), an LSST level 2 milestone pertaining to the Data Management Subsystem.
}


\maketitle

\section{Introduction}
\label{sect:intro}


\subsection{Objectives}
\label{sect:objectives}

This milestone verify the DAQ network from the summit to the base
ensuring that systems at the base can receive communications from the
DAQ.~~



\subsection{System Overview}
\label{sect:systemoverview}

This milestone will take simulated data from the DAQ at the Summit and
use the DWDM network environment to place data on DM machines at Base
Data Center (BDC), essentially extending the DAQ extended network to the
BDC. The test machines at the BDC will be a L1 handoff environment and a
single forwarder\\[2\baselineskip]

\subsection{Applicable Documents}\label{applicable-documents}

\citeds{LDM-294} Data Management Organization and Management\\
\citeds{LDM-503} DM Test Plan\\
\citeds{LDM-148} Data Management System Design


\subsection{Document Overview}
\label{sect:docoverview}

This document was generated from Jira, obtaining the relevant information from the 
\href{https://jira.lsstcorp.org/secure/Tests.jspa#/testPlan/LVV-P54}{LVV-P54}
~Jira Test Plan and related Test Cycles (
  \href{https://jira.lsstcorp.org/secure/Tests.jspa#/testCycle/LVV-C107}{LVV-C107}
).

Section \ref{sect:intro} provides an overview of the test campaign, the system under test (\product{}), the applicable documentation, and explains how this document is organized.
Section \ref{sect:configuration}  describes the configuration used for this test.
Section \ref{sect:personnel} describes the necessary roles and lists the individuals assigned to them.
%Section \ref{sect:plannedtestactivities} provides the list of planned test cycles and test cases, including all relevant information that fully describes the test campaign.

Section \ref{sect:overview} provides a summary of the test results, including an overview in Table \ref{table:summary}, an overall assessment statement and suggestions for possible improvements.
Section \ref{sect:detailedtestresults} provides detailed results for each step in each test case.

The current status of test plan LVV-P54 in Jira is \textbf{ Draft }.

\subsection{References}
\label{sect:references}
\renewcommand{\refname}{}
\bibliography{lsst,refs,books,refs_ads}
\section{Test Configuration}
\label{sect:configuration}

\subsection{Data Collection}

  Observing is not required for this test campaign.

\subsection{Verification Environment}
\label{sect:hwconf}
  DAQ machines at the summit and forwarder machine at the base data center
(BDC) on networks extending the DAQ network from the summit to the base
and BDC 10GigE networks. ~


  \subsection{Entry Criteria}
  DAQ at the summit and test machines at BDC all have been installed and
in working order with all networks configured.~~


  \subsection{Exit Criteria}
  Files and other communications can take place from the summit machines
and the BDC systems for the DAQ network.~~



\newpage
\section{Personnel}
\label{sect:personnel}

The following personnel are involved in this test activity:

\begin{itemize}
\item Test Plan (LVV-P54) owner: 
\item Test Cycles:
\begin{itemize}
  \item LVV-C107 owner: 
    Michelle Butler
  \begin{itemize}
    \item Test case \href{https://jira.lsstcorp.org/secure/Tests.jspa#/testCase/LVV-T1550}{LVV-T1550} tester: 
  \end{itemize}
\end{itemize}
\item Additional Test Personnel involved:
  \begin{itemize}
    \item Test case \href{https://jira.lsstcorp.org/secure/Tests.jspa#/testCase/LVV-T1550}{LVV-T1550}: 
  \end{itemize}
\end{itemize}

\newpage

\section{Overview of the Test Results}
\label{sect:overview}

\subsection{Summary}
\label{sect:summarytable}

\begin{longtable}{p{0.12\textwidth}p{0.2\textwidth}p{0.56\textwidth}p{0.12\textwidth}}
\toprule

  \multicolumn{3}{c}{ Test Cycle {\bf LVV-C107: LDM-503-10 DAQ verification
 }} \\\hline

  {\bf \footnotesize test case} & {\bf \footnotesize status} & {\bf \footnotesize comment} & {\bf \footnotesize issues} \\\toprule

    \href{https://jira.lsstcorp.org/secure/Tests.jspa#/testCase/LVV-T1550}{LVV-T1550}
    & Not Executed &  &
    \\\hline

\caption{Test Results Summary}
\label{table:summary}
\end{longtable}

\subsection{Overall Assessment}
\label{sect:overallassessment}

Not yet available.

\subsection{Recommended Improvements}
\label{sect:recommendations}

Not yet available.

\newpage
\section{Detailed Test Results}
\label{sect:detailedtestresults}


  \subsection{Test Cycle LVV-C107 }

Open test cycle {\it \href{https://jira.lsstcorp.org/secure/Tests.jspa#/testrun/LVV-C107}{LDM-503-10 DAQ verification
}} in Jira.

  LDM-503-10 DAQ verification
\\
  Status: Not Executed

  Verify that the BDC systems on the~ DAQ DWDM network can communicate
with the Summit DAQ system. ~\\[2\baselineskip]


  \subsubsection{Software Version/Baseline}
    Not provided.

  \subsubsection{Configuration}
    Not provided.

  \subsubsection{Test Cases in LVV-C107 Test Cycle}


    \paragraph{Test Case LVV-T1550 - LDM-503-10 DAQ Validation
 }\mbox{}\\

Open  \href{https://jira.lsstcorp.org/secure/Tests.jspa#/testCase/LVV-T1550}{\textit{ LVV-T1550 } }
test case in Jira.

    Verify that the DAQ can talk to test machines at the BDC through the
DWDM network.~


    \textbf{ Preconditions}:\\
    DAQ at the Summit and machines on networks at the base.~~


    Execution status: {\bf Not Executed }

    Final comment:\\


    Detailed step results:

    \begin{longtable}{p{1cm}p{2cm}p{13cm}}
    \hline
    {Step} & \multicolumn{2}{c}{Description, Results and Status}\\ \hline
      1 & Description &

      \begin{minipage}[t]{13cm}{\footnotesize
      have DAQ produce image at the summit~

      \vspace{\dp0}
      } \end{minipage} \\
      \\ \cdashline{2-3}


      & Expected Result &

      \begin{minipage}[t]{13cm}{\footnotesize
      Image on At-archiver~

      \vspace{\dp0}
      } \end{minipage} \\
      \\ \cdashline{2-3}

      & \begin{minipage}[t]{2cm}{Actual\\ Result}\end{minipage}   & 
      \begin{minipage}[t]{13cm}{\footnotesize
      
      \vspace{\dp0}
      } \end{minipage} \\
      \\ \cdashline{2-3}


      & Status          & Not Executed \\ \hline

      2 & Description &

      \begin{minipage}[t]{13cm}{\footnotesize
      Have the at-archiver machine at the summit BBCP a copy of the file to
the L1-handoff machine at the base and ingest the file to the butler~

      \vspace{\dp0}
      } \end{minipage} \\
      \\ \cdashline{2-3}


      & Expected Result &

      \begin{minipage}[t]{13cm}{\footnotesize
      L1-handoff machine can see the file and ingest it into ButlerG3.~~

      \vspace{\dp0}
      } \end{minipage} \\
      \\ \cdashline{2-3}

      & \begin{minipage}[t]{2cm}{Actual\\ Result}\end{minipage}   & 
      \begin{minipage}[t]{13cm}{\footnotesize
      
      \vspace{\dp0}
      } \end{minipage} \\
      \\ \cdashline{2-3}


      & Status          & Not Executed \\ \hline

      3 & Description &

      \begin{minipage}[t]{13cm}{\footnotesize
      Send message back through SAL that the file was ingested~

      \vspace{\dp0}
      } \end{minipage} \\
      \\ \cdashline{2-3}


      & Expected Result &

      \begin{minipage}[t]{13cm}{\footnotesize
      This will make sure that the machines at the base can see and update SAL
messages from the summit. ~\\[2\baselineskip]

      \vspace{\dp0}
      } \end{minipage} \\
      \\ \cdashline{2-3}

      & \begin{minipage}[t]{2cm}{Actual\\ Result}\end{minipage}   & 
      \begin{minipage}[t]{13cm}{\footnotesize
      
      \vspace{\dp0}
      } \end{minipage} \\
      \\ \cdashline{2-3}


      & Status          & Not Executed \\ \hline

    \end{longtable}


\newpage
\appendix
%Make sure lsst-texmf/bin/generateAcronyms.py is in your path
\section{Acronyms used in this document}\label{sec:acronyms}
\addtocounter{table}{-1}
\begin{longtable}{p{0.145\textwidth}p{0.8\textwidth}}\hline
\textbf{Acronym} & \textbf{Description}  \\\hline

BDC &  Base Data Center \\\hline
Center & An entity managed by AURA that is responsible for execution of a federally funded project \\\hline
DAQ & Data Acquisition System \\\hline
DM & Data Management \\\hline
DMTN & DM Technical Note \\\hline
DWDM & Dense Wave Division Multiplex \\\hline
LDM & LSST Data Management (Document Handle) \\\hline
LSST & Large Synoptic Survey Telescope \\\hline
Subsystem & A set of elements comprising a system within the larger LSST system that is responsible for a key technical deliverable of the project \\\hline
Summit & The site on the Cerro Pachón, Chile mountaintop where the LSST observatory, support facilities, and infrastructure will be built \\\hline
Validation & A process of confirming that the delivered system will provide its desired functionality; overall, a validation process includes the evaluation, integration, and test activities carried out at the system level to ensure that the final developed system satisfies the intent and performance of that system in operations \\\hline
\end{longtable}


\end{document}
